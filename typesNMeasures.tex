\chapter{Types and measures}
\section{Unit of Measure}
\label{sec:measures}
 F\# allows for assigning \idx{unit of measure} to the following types,
\begin{quote}
  \mbox{\lstinline{sbyte},}
  \mbox{\lstinline{int},}
  \mbox{\lstinline{int16},}
  \mbox{\lstinline{int32},}
  \mbox{\lstinline{int64},}
  \mbox{\lstinline{single},}
  \mbox{\lstinline{float32},}
  \mbox{\lstinline{float},} and
  \mbox{\lstinline{decimal}.}
\end{quote}
by using the syntax,
%
\begin{lstlisting}[language=EBNF]
"[<Measure>] type" unit-name [ "=" unit-expr ]
\end{lstlisting}
%
and then use \lstinline[language=EBNF]|"<" unit-name ">"| as suffix for literals. E.g., defining unit of measure 'm' and 's', then we can make calculations like,
%
\begin{lstlisting}[language=fsharp,caption={fsharpi, floating point and integer numbers may be assigned unit of measures.}]
> [<Measure>] type m
- [<Measure>] type s 
- let a = 3<m/s^2>
- let b = a * 10<s>
- let c = 4 * b;;

[<Measure>]
type m
[<Measure>]
type s
val a : int<m/s ^ 2> = 3
val b : int<m/s> = 30
val c : int<m/s> = 120
\end{lstlisting}
However, if we mixup unit of measures under addition, then we get an error,
%
\begin{lstlisting}[language=fsharp,caption={fsharpi, unit of measures adds an extra layer of types for syntax checking at compile time.}]
> [<Measure>] type m 
- [<Measure>] type s 
- let a = 1<m>
- let b = 1<s>
- let c = a + b;;

  let c = a + b;;
  ------------^

/Users/sporring/repositories/fsharpNotes/stdin(63,13): error FS0001: The unit of measure 's' does not match the unit of measure 'm'
\end{lstlisting}
Unit of measures allow for \lexeme{*}, \lexeme{/}, and \lexeme{^}\jon{Spec-4.0: this notation is inconsistent with \texttt{**} for float exponentiation.} for multiplication, division and exponentiation. Values with units can be casted to \idx{unit-less} values by casting, and back again by multiplication as,
%
\begin{lstlisting}[language=fsharp,caption={fsharpi, typecasting unit of measures.}]
> [<Measure>] type m      
- let a = 2<m>            
- let b = int a           
- let c = b * 1<m>;;

[<Measure>]
type m
val a : int<m> = 2
val b : int = 2
val c : int<m> = 2
\end{lstlisting}
Compound symbols can be declared as,
%
\begin{lstlisting}[language=fsharp,caption={fsharpi, aggregated unit of measures.}]
> [<Measure>] type s         
- [<Measure>] type m
- [<Measure>] type kg
- [<Measure>] type N = kg * m / s^2;;

[<Measure>]
type s
[<Measure>]
type m
[<Measure>]
type kg
[<Measure>]
type N = kg m/s ^ 2
\end{lstlisting}
For fans of the metric system there is the International System of Units, and these are built-in in \lstinline|Microsoft.FSharp.Data.UnitSystems.SI.UnitSymbols| and give in \Cref{tab:siUnits}.
\begin{table}
  \centering
  \begin{tabularx}{0.75\linewidth}{|l|X|}
    \hline
    Unit & Description \\
    \hline
    \lstinline|A| & Ampere, unit of electric current.\\
    \lstinline|Bq|&Becquerel, unit of radioactivity.\\
    \lstinline|C|&Coulomb, unit of electric charge, amount of electricity.\\
    \lstinline|cd|&Candela, unit of luminous intensity.\\
    \lstinline|F|&Farad, unit of capacitance.\\
    \lstinline|Gy|&Gray, unit of an absorbed dose of radiation.\\
    \lstinline|H|&Henry, unit of inductance.\\
    \lstinline|Hz|&Hertz, unit of frequency.\\
    \lstinline|J|&Joule, unit of energy, work, amount of heat.\\
    \lstinline|K|&Kelvin, unit of thermodynamic (absolute) temperature.\\
    \lstinline|kat|&Katal, unit of catalytic activity.\\
    \lstinline|kg|&Kilogram, unit of mass.\\
    \lstinline|lm|&Lumen, unit of luminous flux.\\
    \lstinline|lx|&Lux, unit of illuminance.\\
    \lstinline|m|&Metre, unit of length.\\
    \lstinline|mol|&Mole, unit of an amount of a substance.\\
    \lstinline|N|&Newton, unit of force.\\
    \lstinline|ohm|&Unitnames.o SI unit of electric resistance.\\
    \lstinline|Pa|&Pascal, unit of pressure, stress.\\
    \lstinline|s|&Second, unit of time.\\
    \lstinline|S|&Siemens, unit of electric conductance.\\
    \lstinline|Sv|&Sievert, unit of dose equivalent.\\
    \lstinline|T|&Tesla, unit of magnetic flux density.\\
    \lstinline|V|&Volt, unit of electric potential difference, electromotive force.\\
    \lstinline|W|&Watt, unit of power, radiant flux.\\
    \lstinline|Wb|&Weber, unit of magnetic flux.\\
    \hline
  \end{tabularx}
  \caption{International System of Units.}
  \label{tab:siUnits}
\end{table}
Hence, using the predefined unit of seconds, we may write,
%
\begin{lstlisting}[language=fsharp,caption={fsharpi, SI unit of measures are built-in.}]
> let a = 10.0<Microsoft.FSharp.Data.UnitSystems.SI.UnitSymbols.s>;;

val a : float<Data.UnitSystems.SI.UnitSymbols.s> = 10.0
\end{lstlisting}
To make the use of these predefined symbols easier, we can import them into the present scope by the \idx{\keyword{open}} keyword,
%
\begin{lstlisting}[language=fsharp,caption={fsharpi, simpler syntax by importing, but beware of namespace pollution.}]
> open Microsoft.FSharp.Data.UnitSystems.SI.UnitSymbols;;
> let a = 10.0<s>;;

val a : float<s> = 10.0
\end{lstlisting}
The \keyword{open} keyword should be used with care, since now all the bindings in \lstinline|Microsoft.FSharp.Data.UnitSystems.SI.UnitSymbols| have been imported into the present scope, and since we most likely do not know, which bindings have been used by the programmers of \lstinline|Microsoft.FSharp.Data.UnitSystems.SI.UnitSymbols|, we do not know which identifiers to avoid, when using \keyword{let} statements. We have obtained, what is known as \idx{namespace pollution}. Read more about namespaces in \Cref{part:structured}.

Using unit of measures is advisable for calculations involving real-world values, since some semantical errors of arithmetic expressions may be discovered by checking the resulting unit of measure.

%%% Local Variables:
%%% TeX-master: "fsharpNotes"
%%% End:
