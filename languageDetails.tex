\chapter{Language Details}

\section{Precedence and associativity}
\begin{table}
  \centering
  \begin{tabularx}{\linewidth}{|>{\hsize=1\hsize\raggedright\arraybackslash}X|>{\hsize=.5\hsize}X|>{\hsize=1.5\hsize}X|}
    \hline
    Operator & Associativity & Description\\
    \hline
    \lstinline[language=ebnf]|ident "<" types ">"| & Left & High-precedence type application\\
    \hline
    \lstinline[language=ebnf]|ident "(" expr ")"| & Left & High-predence application\\
    \hline
    \lstinline[language=ebnf]|"."| & Left & \\
    \hline
    \lstinline[language=ebnf]|prefixOp| & Left & All prefix operators\\
    \hline
    \lstinline[language=ebnf]|"|" rule| & Left & Pattern matching rule\\
    \hline
    \mbox{\lstinline[language=ebnf]|ident expr|,} \mbox{\lstinline[language=ebnf]|"lazy'' expr|,} \mbox{\lstinline[language=ebnf]|"assert'' epxr|} & Left & \\
    \hline
    \lstinline[language=ebnf]|"**" {opChar}| & Right & Exponent like\\
    \hline
    \mbox{\lstinline[language=ebnf]|"*"  {opChar}|,} \mbox{\lstinline[language=ebnf]|"/"  {opChar}|,} \mbox{\lstinline[language=ebnf]|"\%"  {opChar}|} & Left & Infix multiplication like\\
     \hline
    \mbox{\lstinline[language=ebnf]|"-"  {opChar}|,} \mbox{\lstinline[language=ebnf]|"+"  {opChar}|} & Left & Infix addition like\\
     \hline
     \lstinline[language=ebnf]|":?''| & None & \\
     \hline
     \lstinline[language=ebnf]|"::''| & Right & \\
     \hline
     \lstinline[language=ebnf]|"^''  {opChar}| & Right & \\
     \hline
    \mbox{\lstinline[language=ebnf]|"!="  {opChar}|,} \mbox{\lstinline[language=ebnf]|"<"  {opChar}|,} \mbox{\lstinline[language=ebnf]|">"  {opChar}|,} \mbox{\lstinline[language=ebnf]|"="|,} \mbox{\lstinline[language=ebnf]!"|"  {opChar}!,} \mbox{\lstinline[language=ebnf]|"\&"  {opChar}|,} \mbox{\lstinline[language=ebnf]|"\$"  {opChar}|} & Left & Infix addition like\\
     \hline
    \mbox{\lstinline[language=ebnf]|":>"|,} \mbox{\lstinline[language=ebnf]|":?>"|} & Right & \\
     \hline
    \mbox{\lstinline[language=ebnf]|"\&"|,} \mbox{\lstinline[language=ebnf]|"\&\&"|} & Left & Boolean and like\\
     \hline
    \mbox{\lstinline[language=ebnf]|"or"|,} \mbox{\lstinline[language=ebnf]!"||"!} & Left & Boolean or like\\
     \hline
     \lstinline[language=ebnf]|","| & None & \\
     \hline
     \lstinline[language=ebnf]|":="| & Right & \\
     \hline
     \lstinline[language=ebnf]|"->"| & Right & \\
     \hline
    \lstinline[language=ebnf]|"if"| & None & \\
     \hline
    \mbox{\lstinline[language=ebnf]|"function"|,} \mbox{\lstinline[language=ebnf]|"fun"|,} \mbox{\lstinline[language=ebnf]|"match"|,} \mbox{\lstinline[language=ebnf]|"try"|}& None & \\
     \hline
     \lstinline[language=ebnf]|"let"| & None & \\
     \hline
     \lstinline[language=ebnf]|";"| & Right & \\
     \hline
     \lstinline[language=ebnf]!"|"! & Left & \\
     \hline
     \lstinline[language=ebnf]|"when"| & Right & \\
     \hline
     \lstinline[language=ebnf]|"as"| & Right & \\
     \hline
  \end{tabularx}
  \caption{Precedence and associativity of operators. Operators in the same row has same precedence. See Listing~\ref{list:infixOrPrefixOperators} for the definition of \lstinline!prefixOp!}
  \label{tab:operatorPrecedence}
\end{table}

\begin{table}
  \centering
  \begin{tabularx}{\linewidth}{|l|l|X|}
    \hline
    Name& Example & Description\\
    \hline
    \lstinline{fst} & \lstinline{fst (1, 2)} &\\
    \hline
    \lstinline{snd} & \lstinline{snd (1, 2)} &\\
    \hline
    \lstinline{failwith} & \lstinline{failwith} &\\
    \hline
    \lstinline{invalidArg} & \lstinline{invalidArg} &\\
    \hline
    \lstinline{raise} & \lstinline{raise} &\\
    \hline
    \lstinline{reraise} & \lstinline{reraise} &\\
    \hline
    \lstinline{ref} & \lstinline{ref} &\\
    \hline
%    \lstinline|(!)| & \lstinline|!| &\\
    \hline
    \lstinline{ceil} & \lstinline{ceil} &\\
    \hline
  \end{tabularx}
  \caption{Built-in functions.}
  \label{tab:simpleBuiltInFct}
\end{table}

\section{Behind the scene}
\jon{I'm not sure, whether it will be a good idea to describe this. Could be used as the umbrella for the specifikation of the program.} When a program is compiled or interpreted the following steps are performed by the system
\begin{enumerate}
\item Decoding
\item Tokenization
\item Lexical Filtering
\item Parsing
\item Importing
\item Checking
\item Elaboration
\item Execution
\end{enumerate}
\dots

\section{Lightweight Syntax}
\dots
\jon{See Lightweight Syntax, Spec-4.0 Chapter 15.1}

%%% Local Variables:
%%% TeX-master: "fsharpNotes"
%%% End:
