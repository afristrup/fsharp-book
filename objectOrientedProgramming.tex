\chapter{Object-oriented programming}
\label{chap:oop}

\idx{Object-oriented programming} is a programming paradigm that focusses on \idx{objects} such as a person, place, thing, event, and concept relevant for the problem. Objects may contain data and code, which in the object-oriented paradigm are called \idx{attributeds} and \idx{methods}. Object-oriented programming is an extension of data types, in the sense that objects contains both data and functions in a similar manner as a module, but object-oriented programming emphasizes the semantic unity of the data and functions. Thus, objects are \idx{models} of real world entities, and object-oriented programming leads to a particular style of programming analysis and design called \idx[object-oriented analysis]{object-oriented analysis and design}\idxs{object-oriented design}. 

An object is a variable of a class type. A class is defined using the \keyword{type} keyword, and there are allways parantheses after the class name. Consider the following problem.
\begin{problem}
  A complex number is a pair of real numbers called the real and the imaginary part and a set of operators. In particular, addition of two complex numbers is the the addition of their real parts and of their imaginary parts. Define a class for complex numbers including the addition operator.
\end{problem}
A solution to this problem is as follows.
%
\fs{complex}{A class implementing complex numbers and the addition operator.}
%

Things to remember: 
\begin{itemize}
\item upcast and downcast \keyword{upcast}, \lexeme{:>},
  \keyword{downcast}, \lexeme{:?>}
\item boxing \lstinline|(box 5) :?> int;;|, see Spec-4.0 chapter
  18.2.6.
\item obj type Spec-4.0 chapter 18.1
\item boxing Spec-4.0 Section 18.2.6
\end{itemize}

\jon{In object-oriented programming: functions and data are
  combined. Contrast the Anemic Domain Model (\url{https://www.martinfowler.com/bliki/AnemicDomainModel.html})}
%%% Local Variables:
%%% TeX-master: "fsharpNotes"
%%% End:
