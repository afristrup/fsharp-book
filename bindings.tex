\chapter{Constants, functions, and variables}
\label{chap:let}
%
In the previous chapter, we saw how to use F\# as a calculator working with literals, operators and built-in functions. To save time and make programs easier to read and debug, it is useful to bind expressions to identifiers either as new constants, functions, or operators. As an example, consider the problem,
\begin{problem}
  For given set constants $a$, $b$, and $c$, solve for $x$ in
  \begin{equation}
  a x^2+bx+c = 0
\end{equation}
\end{problem}
To solve for $x$ we use the quadratic formula from elementary algebra,
\begin{equation}
  x = \frac{-b\pm\sqrt{b^2-4ac}}{2a},
\end{equation}
which gives the general solution for any values of the coefficients. Here, we will assume a positive discriminant, $b^2-4ac>0$. In order to write a program, where the code may be reused later, we define a function \mbox{\lstinline!discriminant : float -> float -> float!}  , that is, a function that takes 3 arguments, \lstinline!a!, \lstinline!b!, and \lstinline!c!, and calculates the discriminant. Details on function definition is given in Section~\ref{sec:functions}. Likewise, we will define functions \mbox{\lstinline!positiveSolution : float -> float -> float!} and \mbox{\lstinline!negativeSolution : float -> float -> float!}, that also takes the polynomial's coefficients as arguments and calculates the solution corresponding to choosing the postive and negative sign for $\pm$ in the equation. Our solution thus looks like Listing~\ref{identifiersExample}.
%
\fs{identifiersExample}{Finding roots for quadratic equations using function name binding.}
%
Here, we have further defined names of values \lstinline!a!, \lstinline!b!, and \lstinline!c! used as input to our functions, and the results of function application is bound to the names \lstinline!d!, \lstinline!xn!, and \lstinline!xp!. The names of functions and values given here are examples of identifiers, and with these, we may reuse the quadratic formulas and calculated values later, while avoiding possible typing mistakes and reducing amount of code, which needs to be debugged.

Before we begin a deeper discussion note that F\# has adheres to two different syntax: regular and \idx[lightweight syntax]{lightweight}. In the regular syntax, newlines and whitespaces are generally ignored, while in lightweight syntax, certain keywords and lexemes may be replaced by specific use of newlines and whitespaces. Lightweight syntax is the most common, but the syntaxes may be mixed, and we will highlight the options, when relevant.

The use of identifiers is central in programming. For F\# not to be confused by built-in functionality, identifiers must follow a specific grammar: An identifier must start with a letter, but can be followed by zero or more of letters, digits, and a range of special characters except SP, LF, and CR (space, line feed, and carriage return). An identifier must not be a keyword or a reserved-keyword listed in Figures~\ref{fig:keywords}.
\begin{figure}
  \begin{description}
  \item[Keywords:]~\\
    \mbox{\lstinline{abstract},} \mbox{\lstinline{and},} \mbox{\lstinline{as},} \mbox{\lstinline{assert},} \mbox{\lstinline{base},} \mbox{\lstinline{begin},} \mbox{\lstinline{class},} \mbox{\lstinline{default},} \mbox{\lstinline{delegate},} \mbox{\lstinline{do},} \mbox{\lstinline{done},} \mbox{\lstinline{downcast},} \mbox{\lstinline{downto},} \mbox{\lstinline{elif},} \mbox{\lstinline{else},} \mbox{\lstinline{end},} \mbox{\lstinline{exception},} \mbox{\lstinline{extern},} \mbox{\lstinline{false},} \mbox{\lstinline{finally},} \mbox{\lstinline{for},} \mbox{\lstinline{fun},} \mbox{\lstinline{function},} \mbox{\lstinline{global},} \mbox{\lstinline{if},} \mbox{\lstinline{in},} \mbox{\lstinline{inherit},} \mbox{\lstinline{inline},} \mbox{\lstinline{interface},} \mbox{\lstinline{internal},} \mbox{\lstinline{lazy},} \mbox{\lstinline{let},} \mbox{\lstinline{match},} \mbox{\lstinline{member},} \mbox{\lstinline{module},} \mbox{\lstinline{mutable},} \mbox{\lstinline{namespace},} \mbox{\lstinline{new},} \mbox{\lstinline{null},} \mbox{\lstinline{of},} \mbox{\lstinline{open},} \mbox{\lstinline{or},} \mbox{\lstinline{override},} \mbox{\lstinline{private},} \mbox{\lstinline{public},} \mbox{\lstinline{rec},} \mbox{\lstinline{return},} \mbox{\lstinline{sig},} \mbox{\lstinline{static},} \mbox{\lstinline{struct},} \mbox{\lstinline{then},} \mbox{\lstinline{to},} \mbox{\lstinline{true},} \mbox{\lstinline{try},} \mbox{\lstinline{type},} \mbox{\lstinline{upcast},} \mbox{\lstinline{use},} \mbox{\lstinline{val},} \mbox{\lstinline{void},} \mbox{\lstinline{when},} \mbox{\lstinline{while},} \mbox{\lstinline{with},} and \mbox{\lstinline{yield}.}
  \item[Reserved keywords for possible future use:]~\\
    \mbox{\lstinline{atomic},} \mbox{\lstinline{break},} \mbox{\lstinline{checked},} \mbox{\lstinline{component},} \mbox{\lstinline{const},} \mbox{\lstinline{constraint},} \mbox{\lstinline{constructor},} \mbox{\lstinline{continue},} \mbox{\lstinline{eager},} \mbox{\lstinline{fixed},} \mbox{\lstinline{fori},} \mbox{\lstinline{functor},} \mbox{\lstinline{include},} \mbox{\lstinline{measure},} \mbox{\lstinline{method},} \mbox{\lstinline{mixin},} \mbox{\lstinline{object},} \mbox{\lstinline{parallel},} \mbox{\lstinline{params},} \mbox{\lstinline{process},} \mbox{\lstinline{protected},} \mbox{\lstinline{pure},} \mbox{\lstinline{recursive},} \mbox{\lstinline{sealed},} \mbox{\lstinline{tailcall},} \mbox{\lstinline{trait},} \mbox{\lstinline{virtual},} and \mbox{\lstinline{volatile}.}
\item[Symbolic keywords:]~\\
  \mbox{\lstinline{let!},} \mbox{\lstinline{use!},} \mbox{\lstinline{do!},} \mbox{\lstinline{yield!},} \mbox{\lstinline{return!},} \mbox{\lstinline{|},} \mbox{\lstinline{->},} \mbox{\lstinline{<-},} \mbox{\lstinline{.},} \mbox{\lstinline{:},} \mbox{\lstinline{(},} \mbox{\lstinline{)},} \mbox{\lstinline{[},} \mbox{\lstinline{]},} \mbox{\lstinline{[<},} \mbox{\lstinline{>]},} \mbox{\lstinline{[|},} \mbox{\lstinline{|]},} \mbox{\lstinline{\{},} \mbox{\lstinline{\}},} \mbox{\lstinline{'},} \mbox{\lstinline{#},} \mbox{\lstinline{:?>},} \mbox{\lstinline{:?},} \mbox{\lstinline{:>},} \mbox{\lstinline{..},} \mbox{\lstinline{::},} \mbox{\lstinline{:=},} \mbox{\lstinline{;;},} \mbox{\lstinline{;},} \mbox{\lstinline{=},} \mbox{\lstinline{_},} \mbox{\lstinline{?},} \mbox{\lstinline{??},} \mbox{\lstinline{(*)},} \mbox{\lstinline{<@},} \mbox{\lstinline{@>},} \mbox{\lstinline{<@@},} and \mbox{\lstinline{@@>}.} 
\item[Reserved symbolic keywords for possible future:] ~\\
  \mbox{\lstinline{\~}} and \mbox{\lstinline{`}}.
  \end{description}
  \caption{List of (possibly future) keywords and symbolic keywords in F\#.}
  \label{fig:keywords}
\end{figure}
An identifier is a name for a constant, an expression, or a type, and it is defined by the following EBNF:
%
\begin{verbatimwrite}{tmp.ebnf}
ident = (letter | "_") {letter | dDigit | specialChar};
longIdent = ident | ident "." longIdent; (*no space around "."*)

longIdentOrOp = [longIdent "."] identOrOp; (*no space around "."*)
identOrOp = 
  ident
  | "(" infixOp | prefixOp ")" 
  | "(*)";

dDigit = "0" | "1" | "2" | "3" | "4" | "5" | "6" | "7" | "8" | "9";
letter = Lu | Ll | Lt | Lm | Lo | Nl; (*e.g. "A", "B" ... and "a", "b", ...*)
specialChar = Pc | Mn | Mc | Cf; (*e.g., "_"*)

codePoint = ?Any unicode codepoint?;
Lu = ?Upper case letters?;
Ll = ?Lower case letters?;
Lt = ?Digraphic letters, with first part uppercase?;
Lm = ?Modifier letters?;
Lo = ?Gender ordinal indicators?;
Nl = ?Letterlike numeric characters?;
Pc = ?Low lines?;
Mn = ?Nonspacing combining marks?;
Mc = ?Spacing combining marks?;
Cf = ?Soft Hyphens?;
\end{verbatimwrite}
\fsebnf{tmp.ebnf}{Identifiers}
%
Thus, examples of identifiers are \lstinline{a}, \lstinline{theCharacter9}, \lstinline{Next_Word}, \lstinline{_tok}. Typically, only letters from the english alphabet are used as \lstinline[language=ebnf]!letter!, and only \lstinline[language=ebnf]!_! is used for \lstinline[language=ebnf]!specialChar!, but the full definition referes to the Unicode general categories described in Appendix~\ref{sec:unicode}, and there are currently 19.345 possible Unicode code points in the \lstinline[language=ebnf]{letter} category and 2.245 possible Unicode code points in the \lstinline[language=ebnf]{specialChar} category.

Expressions are a central concept in F\#. An expression can be a mathematical expression, such as $3*5$, a function application, such as $f 3$, and many other things. Central in this chapter is the binding of values and functions to identifiers, which is done with the keyword \keyword{let}, using the following simplified syntax, e.g., \lstinline!let a = 1.0!.

Expressions has an enormous variety in how they may be written, we will in this book gradually work through some of the more important facets. For this we will extend the EBNF notation with ellipses: \lstinline[language=ebnf]{...}, to denote that what is shown is part of the complete EBNF production rule. E.g., the part of expressions, we will discuss in this chapter is specified in EBNF by,\spec{Spec-4.0 Section 6.6, function-or-value-defns, possible Mono deviation from specification: let rec functionDefn and functionDefn requires newline before and.} \spec{Spec-4.0 does not seem to mention the special identifier '\_', which discards values in 'let' statements.}
%
\begin{verbatimwrite}{tmp.ebnf}
expr = ...
  | expr ":" type (*type annotation*)
  | expr ";" expr (*sequence of expressions*)
  | "let" valueDefn "in" expr (*binding a value or variable*)
  | "let" ["rec"] functionDefn "in" expr (*binding a function or operator*)
  | "fun" argumentPats "->" expr (*anonymous function*)
  | expr "<-" expr (*assignment*)
 
type = ...
  | longIdent (*named such as "int"*)

valueDefn = ["mutable"] pat "=" expr;

pat = ...
  | "_" (*wildcard*)
  | ident (*named*)
  | pat ":" type (*type constraint*)
  | "(" pat ")" (*parenthesized*)

functionDefn = identOrOp argumentPats [":" type] "=" expr;
argumentPats = pat | pat argumentPats;
\end{verbatimwrite}
\fsebnf{tmp.ebnf}{Simple expressions.}
%
In the following sections, we will work through this bit by bit.

\section{Values}
Binding identifiers to literals or expressions that are evaluated to be values, is called value-binding, and examples are \lstinline!let a = 3.0! and \lstinline!let b = cos 0.9!. On EBNF the simplified syntax,
%
\begin{verbatimwrite}{tmp.ebnf}
expr = ...
  | "let" valueDefn "in" expr (*binding a value or variable*)
\end{verbatimwrite}
\fsebnf{tmp.ebnf}{Value binding expression.}
%
The \keyword{let} bindings defines relations between patterns \lstinline[language=ebnf]!pat! and expressions \lstinline[language=ebnf]!expr! for many different purposes. Most often the pattern is an identifier \lstinline[language=ebnf]!ident!, which \idx{\keyword{let}} defines to be an alias of the expression \lstinline[language=ebnf]!expr!. The pattern may also be defined to have specific type using the \idx{\lexeme{:}} lexeme and a named \lstinline[language=ebnf]!type!. The \lexeme{\_} pattern is called the \idx{wild card} pattern and, when it is in the value-binding, then the expression is evaluated but the result is discarded. The binding may be mutable as indicated by the keyword \idx{\keyword{mutable}}, which will be discussed in Section~\ref{sec:mutableValues}, and the binding holds \idx{lexically} for the last expression as indicated by the \idx{\keyword{in}} keyword. For example, letting the identifier \lstinline!p! be bound to the value \lstinline!2.0! and using it in an expression is done as follows,
%
\fs{letValue}{The identifier \lstinline!p! is used in the expression following the \keyword{in} keyword.}
%
F\# will ignore most newlines between lexemes, i.e., the above is equivalent to writing,
%
\fs{letValueLF}{Newlines after \keyword{in} make the program easier to read.}
%
F\# also allows for an alternative notation called \idx{lightweight syntax}, where e.g., the \keyword{in} keyword is replaced with a newline, and the expression starts on the next line at the same column as \keyword{let} starts in, i.e., the above is equivalent to
%
\fs{letValueLightWeight}{Lightweight syntax does not require the \keyword{in} keyword, but expression must be aligned with the \keyword{let} keyword.}
%
The same expression in interactive mode will also respond the inferred types, e.g.,
%
\fso{letValueLightWeightTypes}{Interactive mode also responds inferred types.}
%
By the \keyword{val} keyword in the line \lstinline!val p : float = 2.0! we see that \lstinline!p! is inferred to be of type \lstinline!float! and bound to the value \lstinline!2.0!. The inference is based on the type of the right-hand-side, which is of type \lstinline!float!.  Identifiers may be defined to have a type using the \lexeme{:} lexeme, but the types on the left-hand-side and right-hand-side of the \lexeme{=} lexeme must be identical. I.e., mixing types gives an error,
%
\fs{letValueTypeError}{Binding error due to type mismatch.}
% \begin{lstlisting}[language=fsharp,caption={fsharpi, binding error due to type mismatch.}]
%   > let a : float = 3;;
%
%   let a : float = 3;;
%   ----------------^
%
% /Users/sporring/repositories/fsharpNotes/stdin(50,17): error FS0001: This expression was expected to have type
%     float    
% but here has type
%     int    
% \end{lstlisting}
Here, the left-hand-side is defined to be an identifier of type float, while the right-hand-side is a literal of type integer.

An expression can be a sequence of expressions separated by the lexeme \lexeme{;}, e.g.,
%
\fs{letValueSequence}{A value-binding for a sequence of expressions.}
%
The lightweight syntax automatically inserts the \lexeme{;} lexeme at newlines, hence using the lightweight syntax the above is the same as,
%
\fs{letValueSequenceLightWeight}{A value-binding for a sequence using lightweight syntax.}
%

A key concept of programming is \idx{scope}. In F\#, the scope of a value-binding is lexically meaning that when F\# determines the value bound to a name, it looks left and upward in the program text for the \keyword{let} statement defining it, e.g.,
%
\fs{letValueScopeLower}{Redefining identifiers is allowed in lower scopes.}
%
F\# also has to option of using dynamic scope, where the value of a binding is defined by when it is used, and this will be discussed in Section~\ref{sec:mutableValues}.

Scopes are given levels, and scopes may be nested, where the nested scope has a level one lower than its parent.\jon{Drawings would be good to describe scope} F\# distinguishes between the top and lower levels, and at the top level in the lightweight syntax, redefining values is not allowed, e.g.,
%
\fs{letValueScopeLowerError}{Redefining identifiers is not allowed in lightweight syntax at top level.}
%
%But using \keyword{begin} and \keyword{end} keywords, we create a \idx{block} which acts as a \idx{nested scope}, and then redefining is allowed, e.g.,
% %
% \fs{letValueScopeBlockAlternative2}{A block has lower scope level, and rebinding is allowed.}
% %
% It is said that the second binding \idx{overshadows} the first.
% Alternatively we may use parentheses to create a block, e.g.,
But using parentheses, we create a \idx{block}, i.e., a \idx{nested scope}, and then redefining is allowed, e.g.,
%
\fs{letValueScopeBlockAlternative3}{A block may be created using parentheses.}
%
In both cases we used indentation, which is good practice, but not required here. Bindings inside are not available outside the nested scope, e.g.,
%
\fs{letValueScopeNestedScope}{Bindings inside a scope are not available outside.}
%
Nesting is a natural part of structuring code, e.g., through function definitions to be discussed in Section~\ref{sec:functions} and flow control structures to be discussed in Chapter~\ref{chap:flow}. Blocking code by nesting is a key concept for making robust code that is easy to use by others without the user necessarily needing to know the details of the inner workings of a block of code.

Defining blocks is useful for controlling the extend of a lexical scope of bindings. For example, adding a second \lstinline!printfn! statement,
%
\fs{letValueScopeBlockProblem}{Overshadowing hides the first binding.}
%
will print the value 4 last bound to the identifier \lstinline!p!, since F\# interprets the above as \lstinline!let p = 3 in let p = 4 in (printfn "%A" p; printfn "%A" p)!. %
Had we intended to print the two different values of \lstinline!p!, the we should have create a block as,
%
\fs{letValueScopeBlock}{Blocks allow for the return to the previous scope.}
%
Here, the lexical scope of \lstinline!let p = 4 in ...! is for the nested scope, which ends at \lexeme{)}, returning to the lexical scope of \lstinline!let p = 3 in ...!. %Alternatively, the \keyword{begin} and \keyword{end} keywords could equally have been used.
%\fs{letValueScopeBlockAlternative}{}

\section{Non-recursive functions}
\label{sec:functions}
A function is a mapping between an input and output domain. A key advantage of using functions, when programming, is that they \idx{encapsulate code} into smaller units, that are easier to debug and may be reused. F\# is a functional first programming language, and offers a number of alternative methods for specifying parameters, which will be discussed in this section. Binding identifiers to functions follows a syntax similar to value-binding,
%
\begin{verbatimwrite}{tmp.ebnf}
expr = ...
  | "let" functionDefn "in" expr (*binding a function or operator*)
\end{verbatimwrite}
\fsebnf{tmp.ebnf}{Function binding expression}
%
Functions may also be recursive, which will be discussed in Chapter~\ref{chap:flow}. An example in interactive mode is,
%
\fso{letFunction}{An example of a binding of an identifier and a function.}
%
and we see that the function is interpreted to have the type \lstinline!val sum : x:float -> y:float -> float!. The \lexeme{->} lexeme means a mapping between sets, in this case floats. The function is also a higher order function, to be discussed in detail below, and here it suffices to think of \lstinline!sum! as a function that takes 2 floats as argument and returns a float.
%, that \lexeme{->} associates to the right, hence \lstinline!x:float -> y:float -> float! is equivalent to \lstinline!x:float -> (y:float -> float)! and thus, \lstinline!sum x! is a function, which gives a function

Not all types need to be declared, just sufficient for F\# to be able to infer the types for the full statement. In the example, one specification is sufficient, and we could just have specified the type of the result,
%
\fse{letFunctionAlterantive}{All types need most often not be specified.}
%
or even just one of the arguments,
%
\fse{letFunctionAlterantive2}{Just one type is often enough for F\# to infer the rest.}
%
In both cases, since the \lstinline|+| \idx{operator} is only defined for \idx[operand]{operands} of the same type, then when the type of either the result, any or both operands are declared, then the type of the remaining follows directly.  As for values, lightweight syntax automatically inserts the keyword \keyword{in} and the lexeme \lexeme{;},
%
\fs{letFunctionLightWeight}{Lightweight syntax for function definitions.}
%

Arguments need not always be inferred to types, but may be of generic type, which F\# prefers, when \idx{type safety} is ensured, e.g.,
%
\fso{functionDeclarationGeneric}{Typesafety implies that a function will work for any type, and hence it is generic.}
%
Here, the function \lstinline{second} does not use the first argument \lstinline{x}, which therefore can be of any type, and which F\# therefore calls \lstinline{'a}, and the type of the second element, \lstinline{y}, can also be of any type and not necessarily the same as \lstinline!x!, so it is called \lstinline!'b!. Finally the result is the same type as \lstinline!y!, whatever it is. This is an example of a \idx{generic function}, since it will work on any type.

A function may contain a sequence of expressions, but must return a value. E.g., the quadratic formula may be written as, 
%
\fs{identifiersExampleAdvance}{A function may contain sequences of expressions.}
%
Here, we used the lightweight syntax, where the \lexeme{=} identifies the start of a nested scope, and F\# identifies the scope by indentation. The amount of space used for indentation is does not matter, but all lines following the first must use the same. The scope ends before the first line with the previous indentation or none. Notice how the last expression is not bound to an identifier, but is the result of the function, i.e., in contrast to many other languages, F\# does not have an explicit keyword for returning values, but requires a final expression, which will be returned to the caller of the function. Note also that since the function \lstinline!discriminant! is defined in the nested scope of \lstinline!solution!, then  \lstinline!discriminant! cannot be called outside \lstinline!solution!, since the scope ends before \lstinline!let a = 1.0!.

\idx[lexical scope]{Lexical scope} and function definitions can be a cause of confusion as the following example shows,\jon{Add a drawing or possibly a spell-out of lexical scope here.}
%
\fs{lexicalScopeNFunction}{Lexical scope means that $f(z) = 3x$ and not $4x$ at the time of calling.}
%
Here, the value-binding for \lstinline!a! is redefined, after it has been used to define a helper function \lstinline!f!. So which value of \lstinline!a! is used, when we later apply \lstinline!f! to an argument? To resolve the confusion, remember that value-binding is lexically defined, i.e., the binding \lstinline!let f z = a * x! uses the value of \lstinline!a!, it has by the ordering of the lines in the script, not dynamically by when \lstinline!f! was called. Hence, \advice{think of lexical scope as substitution of an identifier with its value or function immediately at the place of definition.} I.e., since \lstinline!a! and \lstinline!3.0! are synonymous in the first lines of the program, then the function \lstinline!f! is really defined as, \lstinline!let f z = 3.0 * x!.

Functions do not need a name, but may be declared as an \idx{anonymous function} using the \keyword{fun} keyword and the \lexeme{->} lexeme,
%
\fs{functionDeclarationAnonymous}{Anonymous functions are functions as values.}
%
Here, a name is bound to an anonymous function, which returns the first of two arguments. The difference to \lstinline!let first x y = x! is that anonymous functions may be treated as values, meaning that they may be used as arguments to other functions, and new values may be reassigned to their identifiers, when mutable, as will be discussed in Section~\ref{sec:mutableValues}. A common use of anonymous functions is as as arguments to other functions, e.g.,
%
\fs{functionDeclarationAnonymousAdvanced}{Anonymous functions are often used as arguments for other functions.}
%
Note that here \lstinline!apply! is given 3 arguments, the function \lstinline!mul! and 2 integers. It is not given the result of \lstinline!mul 3 6!, since that would not match the definition of \lstinline!apply!. \advice{Anonymous functions and functions as arguments are powerfull concepts, but tend to make programs harder to read, and their use should be limited.}

Functions may be declared from other functions
%\fs{functionDeclarationTupleCurrying}{}
%For functions of more than 1 argument, there exists a short notation, which is called \idx{currying} in tribute of Haskell Curry,
%
\fs{functionDeclarationCurrying}{A function can be defined as a subset of another by Currying.}
%
Here, \lstinline{mul 2.0} is a partial specification of the function \lstinline{mul x y}, where the first argument is fixed, and hence, \lstinline{timesTwo} is a function of 1 argument being the second argument of \lstinline{mul}. This notation is called \idx{currying} in tribute of Haskell Curry, and Currying is often used in functional programming, but generally \advice{currying should be used carefully, since currying may seriously reduce readability of code.}

A \idx{procedure} is a generalisation of the concept of functions, and in contrast to functions procedures need not return values,
%
\fs{procedure}{A procedure is a function that has no return value, and in F\# returns \lexeme{()}.}
In F\# this is automatically given the unit type as return value. Procedural thinking is useful for \idx{encapsulation} of scripts, but is prone to \idx{side-effects} and should be minimized by being replaced by functional thinking. More on side-effects in Section~\ref{sec:mutableValues}.
%An example, we've already seen is the \texttt{printfn}, which is used to print text on the console, but does not return a value. Coincidentally, since the console is a state, printing to it is a side-effect. Above we examined 
%\begin{fse}
%   let updateFactor factor = 
%    factor := 2
%\end{fse}
%\fse{mutableAssignReturnSideEffectStump}{}
%which also does not have a return value. 
\advice{Procedural thinking is useful for encapsulation, but is prone to side-effects and should be minimized by being replaced by functional thinking.}

\section{User-defined operators}
\label{sec:operators}
Operators are functions, and in F\#, the infix multiplication operator \lstinline!+! is equivalent to the function \lstinline!(+)!, e.g.,
%
\fs{addOperatorNFunction}{Operators have function equivalents.}
%
All operator has this option, and you may redefine them and define your own operators, but in F\# names of user-defined operators are limited by the following simplified EBNF:
\begin{verbatimwrite}{tmp.ebnf}
infixOrPrefixOp = "+" | "-" | "+." | "-." | "%" | "&" | "&&";
prefixOp = infixOrPrefixOp | "~" {"~"} | "!" {opChar} - "!=";
infixOp = 
  {"."} (
    infixOrPrefixOp  
    | "-" {opChar} 
    | "+" {opChar} 
    | "||" 
    | "<" {opChar} 
    | ">" {opChar} 
    | "="  
    | " |" {opChar} 
    | "&" {opChar} 
    | "^" {opChar} 
    | "*" {opChar} 
    | "/" {opChar} 
    | "%" {opChar} 
    | "!=" )
  | ":=" | "::" | "$" | "?";
opChar = 
  "!" | "%" | "&" | "*" | "+" | "-" | ". " | "/" 
  | "<" | "=" | ">" | "@" | "^" | "|" | "~";
\end{verbatimwrite}
\fsebnf{tmp.ebnf}{Grammar for infix and prefix lexemes\label{list:infixOrPrefixOperators}}
%op = opChar {opChar}
%identOrOp = ident | "(" op ")"
%$
The precedence rules and associativity of user-defined operators follows the rules for which they share prefixes with built-in rules, see Table~\ref{tab:operatorPrecedence}. E.g., \lstinline!.*!, \lstinline!+++!, and \lstinline!<+! are valid operator names for infix operators, they have precedence as ordered, and their associativity are all left. Using \lstinline!~! as the first character in the definition of an operator makes the operator unary and will not be part of the name. Examples of definitions and use of operators are,
%
\fs{operatorDefinitions}{Operators may be (re)defined by their function equivalent.}
%
Operators beginning with \lstinline!*! must use a space in its definition, \lstinline!( *! in order for it not to be confused with the beginning of a comment \lstinline!(*!, see Chapter~\ref{chap:documentation} for more on comments in code.

Beware, redefining existing operators lexically redefines all future uses of the operators for all types, hence \advice{it is not a good idea to redefine operators, but better to define new.}\spec{It seems there is a bug in mono: \lstinline{let (~+) x = x+1 in printfn "\%A" +1;;} prints 1 and not 2.} In Chapter~\ref{chap:oop} we will discuss how to define type specific operators including prefix operators. 

\section{The Printf function}
\label{sec:printf}
A common way to output information to the console is to use one of the family of \idx{\lstinline{printf}} commands. These functions are special, since they take a variable number of arguments, and the number is decided by the first - the format string,
%
\begin{verbatimwrite}{tmp.ebnf}
"printf" formatString {ident}
\end{verbatimwrite}
\fsebnf{tmp.ebnf}{\lstinline{printf} statement.}
%
where a \lstinline[language=ebnf]!formatString! is a string (simple or verbatim) with placeholders. The function \lstinline[language=ebnf]!printf! prints \lstinline[language=ebnf]!formatString! to the console, where all \lstinline[language=ebnf]!placeholder! has been replaced by the value of the corresponding argument formatted as specified, e.g., in \mbox{\lstinline!printfn "1 2 \%d" 3!} the \lstinline[language=ebnf]!formatString! is \mbox{\lstinline!"1 2 \%d"!}, and the placeholder is \lstinline{%d}, %
and the \lstinline{printf} replaced the placeholder with the value of the corresponding argument, and the result is printed to the console, in this case \mbox{\lstinline!1 2 3!}. Possible formats for the placeholder are,
%
%["0"] ["+"] ["-"] [SP] ("*" | [dInt] ["." dInt]) [placeholderType]
\begin{verbatimwrite}{tmp.ebnf}
placeholder = "%%" | ("%" [flags] [width] ["." precision] specifier) (* No spaces between rules *)
flags = ["0"] ["+"] [SP] (* No spaces between rules *)
width = ["-"] ("*" | [dInt]) (* No spaces between rules *)
specifier = "b" | "d" | "i" | "u" | "x" | "X" | "o" | "e" | "E" | "f" | "F" | "g" | "G" | "M" | "O" | "A" | "a" | "t"
\end{verbatimwrite}
\fsebnf{tmp.ebnf}{Placeholders in \lstinline[language=ebnf]{formatString} for \lstinline!printf! functions.}
%
There are specifiers for all the basic types and more as elaborated in Table~\ref{tab:printfPlaceholder}.
\begin{table}
  \centering
  \rowcolors{2}{oddRowColor}{evenRowColor}
  \begin{tabularx}{\linewidth}{|l|l|X|}
    \hline
    \rowcolor{headerRowColor} Specifier& Type&Description\\
    \hline
    \lstinline!\%b!&\lstinline!bool!&Replaces with boolean value\\
    \hline
    \lstinline!\%s!&\lstinline!string!&\\
    \hline
    \lstinline!\%c!&\lstinline!char!&\\
    \hline
    \mbox{\lstinline!\%d!,} \mbox{\lstinline!\%i!}&basic integer&\\
    \hline
    \lstinline!\%u!&basic unsigned integers&\\
    \hline
    \lstinline!\%x!&basic integer&formatted as unsigned hexadecimal with lower case letters\\
    \hline
    \lstinline!\%X!&basic integer&formatted as unsigned hexadecimal with upper case letters\\
    \hline
    \lstinline!\%o!&basic integer&formatted as unsigned octal integer\\
    \hline
    \mbox{\lstinline!\%f!,} \mbox{\lstinline!\%F!,} &basic floats&formatted on decimal form\\
    \hline
    \mbox{\lstinline!\%e!,} \mbox{\lstinline!\%E!,} &basic floats&formatted on scientific form. Lower case uses "e" while upper case uses "E'' in the formatting.\\
    \hline
    \mbox{\lstinline!\%g!,} \mbox{\lstinline!\%G!,} &basic floats&formatted on the shortest of the corresponding decimal or scientific form.\\
    \hline
    \lstinline!\%M!&decimal&\\
    \hline
    \lstinline!\%O!&Objects \lstinline!ToString! method&\\
    \hline
    \lstinline!\%A!&any built-in types&Formatted as a literal type\\
    \hline
    \lstinline!\%a!&\lstinline[language=ebnf]!Printf.TextWriterFormat ->'a -> ()!&\\
    \hline
    \lstinline!\%t!&\lstinline[language=ebnf]!(Printf.TextWriterFormat -> ()!&\\
    \hline
  \end{tabularx}
  \caption{Printf placeholder string}
  \label{tab:printfPlaceholder}
\end{table}
The placeholder can be given a specified with, either by setting a specific integer, or using the \lstinline!*! character, indicating that the with is given as an argument prior to the replacement value. Default is for the value to be right justified in the field, but left justification can be specified by the \lstinline!-! character. For number types, you can specify their format by: \lstinline!"0"! for padding the number with zeros to the left, when right justifying the number; \lstinline!"+"! to explicitly show a plus sign for positive numbers; \lstinline!SP! to enforce a space, where there otherwise would be a plus sign for positive numbers. For floating point numbers, the precision integer specifies the number of digits displayed of the fractional part. Examples of some of these combinations are,
%
\fs{printfExample}{Examples of printf and some of its formatting options.}
%
Not all combinations of flags and identifier types are supported, e.g., strings cannot have number of integers after the decimal specified.\spec{Mono seems to have a bug, \lstinline{printfn "\%.1g"   3.13;;} and \lstinline{printfn "\%.1f"   3.13;;} produces different number of digits.}\spec{Spec-4.0 \lstinline{\%s} and \lstinline{\%b} are missing in Section 3.1.16.}
The placeholder types \lstinline{"%A"}, \lstinline{"%a"}, and \lstinline{"%t"} are special for F\#, examples of their use are,
%
\fs{printfExampleAdvance}{Custom format functions may be used to specialize output.}
%
The \lstinline!%A! is special in that all built-in types including tuples, lists, and arrays to be discussed in Chapter~\ref{chap:lists} can be printed using this formatting string, but notice that the formatting performed includes the named literal string. The two formatting strings \lstinline!%t! and \lstinline!%a! are options for user-customizing the formatting, and will not be discussed further.

Beware, \lstinline[language=ebnf]!formatString! is not a \lstinline!string! but a \lstinline!Printf.TextWriterFormat!, so to predefine a \lstinline[language=ebnf]!formatString! as, e.g., \mbox{\lstinline{let str = "hello \%s" in printf str "world"}} will be a type error.

The family of \lstinline!printf! is shown in Table~\ref{tab:printfFamily}.\idxss{\lstinline{stdout}}\idxss{\lstinline{stderr}}\idxss{\lstinline{printf}}\idxss{\lstinline{printfn}}\idxss{\lstinline{fprintf}}\idxss{\lstinline{fprintfn}}\idxss{\lstinline{eprintf}}\idxss{\lstinline{eprintfn}}\idxss{\lstinline{sprintf}}\idxss{\lstinline{failwithf}}
\begin{table}
  \centering
  \rowcolors{2}{oddRowColor}{evenRowColor}
  \begin{tabularx}{\linewidth}{|l|l|X|}
    \rowcolor{headerRowColor} Function & Example & Description\\
    \hline
    \lstinline!printf! & \lstinline!printf "\%d apples" 3! & Prints to the console, i.e., \lstinline!stdout!\\
    \lstinline!printfn! &  & as \lstinline!printf! and adds a newline.\\
    \hline
    \lstinline!fprintf! & \lstinline!fprintf stream "\%d apples" 3! & Prints to a stream, e.g., \lstinline!stderr! and \lstinline!stdout!, which would be the same as \lstinline!printf! and \lstinline!eprintf!.\\
    \lstinline!fprintfn! & & as \lstinline!fprintf! but with added newline.\\
    \hline
    \lstinline!eprintf! & \lstinline!eprintf "\%d apples" 3! & Print to \lstinline!stderr!\\
    \lstinline!eprintfn! & & as \lstinline!eprintf! but with added newline.\\
    \hline
    \lstinline!sprintf! & \lstinline!printf "\%d apples" 3! & Return printed string\\
    \hline
    \lstinline!failwithf! & \lstinline!failwithf "\%d failed apples" 3! & prints to a string and used for raising an exception.\\
    \hline
  \end{tabularx}
  \caption{The family of printf functions.}
  \label{tab:printfFamily}
\end{table}
The function \lstinline!fprintf! prints to a stream, e.g., \lstinline!stderr! and \lstinline!stdout!, of type \lstinline!System.IO.TextWriter!. Streams will be discussed in further detail in Chapter~\ref{chap:IO}. The function \lstinline!failwithf! is used with exceptions, see Chapter~\ref{chap:exceptions} for more details. The function has a number of possible return value types, and for testing the \idx{\lstinline{ignore}} function ignores it all, e.g., \mbox{\lstinline!ignore (failwithf "\%d failed apples" 3)!}
\spec{Mono: \lstinline{bprintf} and \lstinline{kprintf} are undefined.}

\section{Variables}
\label{sec:mutableValues}
The \keyword{mutable} in \keyword{let} bindings means that the identifier may be rebound to a new value using the \idx{\lexeme{<-}} lexeme with the following syntax,\jon{Discussion on heap and stack should be added here.}
%
\begin{verbatimwrite}{tmp.ebnf}
expr = ...
  | expr "<-" expr (*assignment*)
\end{verbatimwrite}
\fsebnf{tmp.ebnf}{Value reassignment for mutable variables.}
%
\idx{Mutable data} is synonymous with the term \idx{variable}. A variable is an area in the computers working memory associated with an identifier and a type, and this area may be read from and written to during program execution. For example,
%
\fs{mutableAssignReassingShort}{A variable is defined and later reassigned a new value.}
%\fs{mutableAssignReassing}{}
%
Here, an area in memory was denoted \lstinline{x}, initially assigned the integer value 5, hence the type was inferred to be \lstinline|int|.  Later, this value of \lstinline{x} was replaced with another integer using the \idx{\lexeme{<-}} lexeme. The \lexeme{<-} lexeme is used to distinguish the assignment from the comparison operator, i.e., if we by mistake had written,
%
\fso{mutableEqual}{It is a common error to mistake \lexeme{=} and \lexeme{<-} lexemes for mutable variables.}
% \begin{lstlisting}[language=fsharp,caption={fsharpi, example of changing the content of a variable.}]
% > let mutable a = 0
% - a = 3;;
%
% val mutable a : int = 0
% val it : bool = false
% \end{lstlisting}
%
then we instead would have obtained the default assignment of the result of the comparison of the content of \lstinline|a| with the integer 3, which is false. However, it is important to note, that when the variable is initially defined, then the '\lexeme{=}' operator must be used, while later reassignments must use the \lexeme{<-} expression.

Assignment type mismatches will result in an error, 
%
\fs{mutableAssignReassingTypeError}{Assignment type mismatching causes a compile time error.}
%
I.e., once the type of an identifier has been declared or inferred, then it cannot be changed.

A typical variable is a counter of type integer, and a typical use of counters is to increment them, for example,
%
\fs{mutableAssignIncrement}{Variable increment is a common use of variables.}
%
Using variables in expressions as opposed to the left-hand-side of an assignment operation, reads the value of the variable. Thus, when using a variable as the return value of a function, then the value is copied from the local scope of the function to the scope from which it is called. E.g.,
%
\fso{mutableAssignReturnVariable}{Returning a mutable variable returns its value.}
%
In the example, we see that the type is a value, and not mutable. 

Variables implement dynamic scope, e.g., in comparison with the lexical scope, where the value of an identifier depends on \emph{where} it is defined, dynamic scope depends on, \emph{when} it is used. E.g., the script in Listing~\ref{lexicalScopeNFunction} defines a function using lexical scope and returns the number \lstinline!6.0!, however, if \lstinline!a! is made \lstinline!mutable!, then the behaviour is different:
%
\fs{dynamicScopeNFunction}{Mutual variables implement dynamics scope rules. Compare with Listing~\ref{lexicalScopeNFunction}.}
%
Here, the respons is \lstinline!8.0!, since the value of \lstinline!a! changed befor the function \lstinline!f! was called.
 
It is possible to work with mutable variables but through a special technique called \idx{encapsulation}. E.g., in the following example the we create a counter as an encapsulated mutable variable,
% 
\fs{mutableAssignIncrementEncapsulation}{Local mutable content can be indirectly accessed outside its scope.}
% 
This works because the line \mbox{\lstinline!let mutable counter = 0!} is only executed once, when the function \lstinline!incr! is defined. This is also an example of a side-effect. \advice{Encapsulation of mutable data is good programming practice, but avoiding mutable data all together is better practice.}

F\# has a variation of mutable variables called \idx{reference cells}. Reference cells have built-in function \lstinline{ref} and operators \lexeme{!} and \lexeme{:=},
%
\fs{refCell}{Reference cells are variants of mutable variables.}
%
That is, the \lstinline{ref} function creates a reference variable, the '\lexeme{!}' and the '\lexeme{:=}' operators reads and writes its value. Reference cells are in some language called pointers, and their use is strongly discouraged, since they may cause \idx{side-effects}, where variable changes are performed across independent scopes. The \lstinline!incr! example in Listing~\ref{mutableAssignIncrementEncapsulation} is an example of a side-effect. Another example is,
%
\fs{mutableAssignReturnSideEffect}{Intertwining independent scopes is typically a bad idea.}
%
In the example, the function \lstinline{updateFactor} changes a variable in the scope of \lstinline{multiplyWithFactor}, which is prone to errors, since the computations are not local at the place of writing, i.e., in \lstinline{multiplyWithFactor}, and if \lstinline{updateFactor} were defined in a library, then the source code may not be available. Better style of programming is,
%
\fs{mutableAssignReturnWithoutSideEffect}{A solution of Listing~\ref{mutableAssignReturnSideEffect} avoiding side-effects.}
%
Here, there can be no doubt in \lstinline{multiplyWithFactor} that the value of \lstinline{a} is changing. Side-effects do have their use, but should in general be avoided at almost all costs, and in general it is advised to \advice{minimize the use of side effects}.

Reference cells gives rise to an effect called \idx{aliasing}, where two or more identifiers refer to the same data as illustrated by the following example:
%
\fs{refCellAliasing}{Aliasing can cause surprising results and should be avoided.}
%
Here, \lstinline!a! is defined as a reference cell, and by defining \lstinline!b! to be equal to \lstinline!a!, we have created an alias. This can be very confusing, since as the example shows, changing the value of \lstinline!b! causes \lstinline!a! to change as well. Aliasing is a variant of side-effects, and \advice{aliasing should be avoided at all costs}.

%%% Local Variables:
%%% TeX-master: "fsharpNotes"
%%% End:
