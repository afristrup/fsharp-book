\chapter{Commonly used character sets}
\label{sec:characterSets}
Letters, digits, symbols and space are the core of how we store data, write programs, and comunicate with computers and each others. These symbols are in short called characters, and represents a mapping between numbers, also known as codes, and a pictorial representation of the character. E.g., the ASCII code for the letter 'A' is 65. These mappings are for short called character sets, and due to differences in natural languages and symbols used across the globe, many different character sets are in use. E.g., the English alphabet contains the letters 'a' to 'z', which is shared by many other European languages, but which have other symbols and accents for example, Danish has further the letters 'æ', 'ø', and 'å'. Many non-european languages have completely different symbols, where Chinese character set is probably the most extreme, where some definitions contains 106,230 different characters albeit only 2,600 are included in the official Chinese language test at highest level.

Presently, the most common character set used is Unicode Transformation Format (UTF), whose most popular encoding schemes are 8-bit (UTF-8) and 16-bit (UTF-16). Many other character sets exists, and many of the later builds on the American Standard Code for Information Interchange (ASCII). The ISO-8859 codes were an intermediate set of character sets that are still in use, but which is greatly inferior to UTF. Here we will briefly give an overview of ASCII, ISO-8859-1 (Latin1), and UTF.

\section{ASCII}
\label{sec:ascii}
The \idx{American Standard Code for Information Interchange} (\idx{ASCII})~\cite{ascii63}, is a 7 bit code tuned for the letters of the english language, numbers, punctuation symbols, control codes and space, see Tables~\ref{tab:ascii} and~\ref{tab:asciiSpecialSymbols}. The first 32 codes are reserved for non-printable control characters to control printers and similar devices or to provide meta-information. The meaning of each control characters is not universally agreed upon.

The code order is known as \idx{ASCIIbetical order}, and it is sometimes used to perform arithmetic on codes, e.g., an upper case letter with code $c$ may be converted to lower case by adding 32 to its code. The ASCIIbetical order also has consequence for sorting, i.e., when sorting characters according to their ASCII code, then 'A' comes before 'a', which comes before the symbol '$\{$'.
\begin{table}
  \centering
  \rowcolors{2}{oddRowColor}{evenRowColor}
  \begin{tabularx}{0.75\textwidth}{|*{17}{>{\centering\arraybackslash}X|}}
    \hline
    \rowcolor{headerRowColor} x0+0x & 00 & 10 & 20 & 30 & 40 & 50 & 60 & 70 \\
    \hline
    00 & NUL & DLE & SP & 0 & @ & P & ` & p \\
    \hline
    01 & SOH & DC1 & ! & 1 & A & Q & a & q \\
    \hline
    02 & STX & DC2 & " & 2 & B & R & b & r \\
    \hline
    03 & ETX & DC3 & \# & 3 & C & S & c & s \\
    \hline
    04 & EOT & DC4 & \$ & 4 & D & T & d & t \\
    \hline
    05 & ENQ & NAK & \% & 5 & E & U & e & u \\
    \hline
    06 & ACK & SYN & \& & 6 & F & V & f & v \\
    \hline
    07 & BEL & ETB & ' & 7 & G & W & g & w \\
    \hline
    08 & BS & CAN & ( & 8 & H & X & h & x \\
    \hline
    09 & HT & EM & ) & 9 & I & Y & i & y \\
    \hline
    0A & LF & SUB & * & : & J & Z & j & z \\
    \hline
    0B & VT & ESC & + & ; & K & [ & k & $\{$\\
    \hline
    0C & FF & FS & , & $<$ & L & \textbackslash & l & $|$\\
    \hline
    0D & CR & GS & $-$ & = & M & ] & m & $\}$\\
    \hline
    0E & SO & RS & . & $>$ & N & \textasciicircum & n & \textasciitilde\\
    \hline
    0F & SI & US & / & ? & O & \_ & o & DEL\\
    \hline
  \end{tabularx}
  \caption{ASCII}
  \label{tab:ascii}
\end{table}
\begin{table}
  \centering
  \rowcolors{2}{oddRowColor}{evenRowColor}
  \begin{tabular}{|l|l|}
    \hline
    \rowcolor{headerRowColor} Code & Description\\
    \hline
    NUL & Null\\
    SOH & Start of heading\\
    STX & Start of text\\
    ETX & End of text\\
    EOT & End of transmission\\
    ENQ & Enquiry\\
    ACK & Acknowledge\\
    BEL & Bell\\
    BS & Backspace\\
    HT & Horizontal tabulation\\
    LF & Line feed\\
    VT & Vertical tabulation\\
    FF & Form feed\\
    CR & Carriage return\\
    SO & Shift out\\
    SI & Shift in\\
    DLE & Data link escape\\
    DC1 & Device control one\\
    DC2 & Device control two\\
    DC3 & Device control three\\
    DC4 & Device control four\\
    NAK & Negative acknowledge\\
    SYN & Synchronous idle\\
    ETB & End of transmission block\\
    CAN & Cancel\\
    EM & End of medium\\
    SUB & Substitute\\
    ESC & Escape\\
    FS & File separator\\
    GS & Group separator\\
    RS & Record separator\\
    US & Unit separator\\
    SP & Space\\
    DEL & Delete\\
    \hline
  \end{tabular}
  \caption{ASCII symbols.}
  \label{tab:asciiSpecialSymbols}
\end{table}

\section{ISO/IEC 8859}
The ISO/IEC 8859 report \url{http://www.iso.org/iso/catalogue_detail?csnumber=28245} defines 10 sets of codes specifying up to 191 codes and graphic characters using 8 bits. Set 1 also known as ISO/IEC 8859-1, Latin alphabet No.\ 1, or \idx{Latin1} covers many European languages and is designed to be compatible with ASCII, such that code for the printable characters in ASCII are the same in ISO 8859-1. In Table~\ref{tab:latin1} is shown the characters above 7e. Codes 00-1f and 7f-9f are undefined in ISO 8859-1. 
\begin{table}
  \centering
  \rowcolors{2}{oddRowColor}{evenRowColor}
  \begin{tabularx}{0.75\textwidth}{|*{17}{>{\centering\arraybackslash}X|}}
    \hline
    \rowcolor{headerRowColor} x0+0x & 80 & 90 & A0 & B0 & C0 & D0 & E0 & F0 \\
    \hline
    00 & & & NBSP & $^\circ$ & \`A & \DH & \`a &\dh\\
    \hline
    01 & & & \textexclamdown & $\pm$ & \'A & \~N & \'a &\~n\\
    \hline
    02 & & & \textcent & $^2$ & \^A & \`O & \^a &\`o\\
    \hline
    03 & & & \pounds & $^3$ & \~A & \'O & \~a &\'o\\
    \hline
    04 & & & \textcurrency & \'{} & \"A & \^O & \"a &\^o\\
    \hline
    05 & & & \textyen & $\mu$ & \r A & \~O & \r a &\~o\\
    \hline
    06 & & & \textbrokenbar & \P & \AE & \"O & \ae &\"o\\
    \hline
    07 & & & \S & \textperiodcentered & \c C & $\times$ & \c c &$\div$\\
    \hline
    08 & & & \"{} & \c\ & \`E & \O & \`e &\o\\
    \hline
    09 & & & \copyright & $^1$ & \'E & \`U & \'e &\`u\\
    \hline
    0a & & & \textordfeminine & \textordmasculine & \^E & \'U & \^e &\'u\\
    \hline
    0b & & & \guillemotleft & \guillemotright & \"E & \^U & \"e &\^u\\
    \hline
    0c & & & $\lnot$ & $\frac14$ & \`I & \"U & \`\i &\"u\\
    \hline
    0d & & & SHY & $\frac12$ & \'I & \'Y & \'\i &\'y\\
    \hline
    0e & & & \textregistered & $\frac34$ & \^I & \TH & \^\i &\th\\
    \hline
    0f & & & \={} & \textquestiondown & \"I & \ss & \"\i &\"y\\
    \hline
  \end{tabularx}
  \caption{ISO-8859-1 (latin1) non-ASCII part. Note that the codes 7f -- 9f are undefined.}
  \label{tab:latin1}
\end{table}
\begin{table}
  \centering
  \rowcolors{2}{oddRowColor}{evenRowColor}
  \begin{tabular}{|l|l|}
    \hline
    \rowcolor{headerRowColor} Code & Description\\
    \hline
    NBSP & Non-breakable space\\
    SHY & Soft hypen\\
    \hline
  \end{tabular}
  \caption{ISO-8859-1 special symbols.}
  \label{tab:latin1SpecialSymbols}
\end{table}

\section{Unicode}
\label{sec:unicode}
Unicode is a character standard defined by the Unicode Consortium, \url{http://unicode.org} as the \idx{Unicode Standard}. Unicode allows for 1,114,112 different codes. Each code is called a \idx{code point}, which represents an abstract character. However, not all abstract characters requires a unit of several code points to be specified. Code points are divided into 17 planes each with $2^{16}=65,536$ code points. Planes are further subdivided into named \idx{blocks}. The first plane is called the \idx{Basic Multilingual plane} and it are the first 128 code points is called the \idx{Basic Latin block} and are identical to ASCII, see Table~\ref{tab:ascii}, and code points 128-255 is called the \idx{Latin-1 Supplement block}, and are identical to the upper range of ISO 8859-1, see Table~\ref{tab:latin1}.  Each code-point has a number of attributes such as the \idx{unicode general category}. Presently more than 128,000 code points covering 135 modern and historic writing systems, and obtained at \url{http://www.unicode.org/Public/UNIDATA/UnicodeData.txt}, which includes the code point, name, and general category.

A unicode code point is an abstraction from the encoding and the graphical representation of a character. A code point is written as``U+'' followed by its hexadecimal number, and for the Basic Multilingual plane 4 digits are used, e.g., the code point with the unique name LATIN CAPITAL LETTER A has the unicode code point is ``U+0041'', and is in this text it is visualized as 'A'. More digits are used for code points of the remaining planes.

The general category is used in grammars to specify valid characters, e.g., in naming identifiers in F\#. Some categories and their letters in the first 256 code points are shown in Table~\ref{tab:generalCategories}.
\begin{table}
  \centering
%  \begin{tabularx}{0.75\linewidth}{|>{\hsize=1\hsize}X|\hsize=1\hsize}X|\hsize=1\hsize}X|}
  \rowcolors{2}{oddRowColor}{evenRowColor}
  \begin{tabularx}{\linewidth}{|>{\hsize=.3\hsize}X*{2}{|>{\hsize=1.35\hsize\raggedright\arraybackslash}X}|}    \hline
    \rowcolor{headerRowColor} General category & Code points & Name \\
    \hline
    Lu& U+0041--U+005A, U+00C0--U+00D6,  U+00D8--U+00DE & Upper case letters\\
    Ll& U+0061--U+007A, U+00B5, U+00DF--U+00F6, U+00F8--U+00FF & Lower case letter\\
    Lt& None & Digraphic letter, with first part uppercase \\
    Lm& None & Modifier letter \\
    Lo& U+00AA, U+00BA & Gender ordinal indicator \\
    Nl& None & Letterlike numeric character \\
    Pc& U+005F & Low line\\
    Mn&None & Nonspacing combining mark \\
    Mc&None & Spacing combining mark\\
    Cf& U+00AD & Soft Hyphen \\
    \hline
  \end{tabularx}
  \caption{Some general categories for the first 256 code points.}
  \label{tab:generalCategories}
\end{table}

To store and retrieve code points, they must be encoded and decoded. A common encoding is \idx{UTF-8}, which encodes code points as 1 to 4 bytes, and which is backward-compatible with ASCII and ISO 8859-1. Hence, in all 3 coding systems the character with code 65 represents the character 'A'. Another popular encoding scheme is \idx{UTF-16}, which encodes characters as 2 or 4 bytes, but which is not backward-compatible with ASCII or ISO 8859-1. UTF-16 is used internally in many compiles, interpreters and operating systems.

%%% Local Variables:
%%% TeX-master: "fsharpNotes"
%%% End: