%%%%%%%%%%%%%%%%%%%%%%preface.tex%%%%%%%%%%%%%%%%%%%%%%%%%%%%%%%%%%%%%%%%%
% sample preface
%
% Use this file as a template for your own input.
%
%%%%%%%%%%%%%%%%%%%%%%%% Springer %%%%%%%%%%%%%%%%%%%%%%%%%%

\preface

This book has been written as an introduction to programming for novice programmers. It is used in the first programming course at the University of Copenhagen's bachelor in computer science program. It has been typeset in \LaTeX, and all programs have been developed and tested in dotnet version 6.0.101

%This book started as a few chapters in 2016 and was to a large extent completed in 2017. This book was developed alongside the course Programmering og Problemløsning (programming and problem solving) and I am very thankful for the positive feedback and suggestions numerous people have given me.  I would particularly like to thank Malthe Sporring for his insightful and detailed comments to every (!) page of this book. I also would like to acknowledge the invaluable feedback from my co-teachers: Torben Mogensen, Martin Elsmann, Christina Lioma; my teaching assistants: Sune Hellfritzsch, Emil Bak, Jesper Erno, Rasmus Johannesson, Jan Rolandsen, Peter Pedersen, Joachim Tilsted Kristensen, Lukas Svarre Engedal, Matthias Brix, Kristian Fogh Nissen, Emil Petersen, Jens Larsen, Emil Bak, Lasse Grønborg, Mads Obitsø, Maurits Pallesen, Tor Skovsgaard, Baldar Ivarsen, Alexander Christensen, Lars-Bo Nielsen, Frederik Schmidt, Lukas Engedal, Jan Rolandsen. And finally, thanks to all the students of our course who have had the patience and endurance to labor and enjoy learning to program using F\#.

\vspace*{1cm}
Jon Sporring\\
Professor, Ph.d.\\
Department of Computer Science,\\
University of Copenhagen\\
\today\\
