\documentclass[fsharpnotes.tex]{subfiles}

\begin{document}
\chapter{Imperative programming}

\section{Introduction}
\idx[imperative programming]{Imperativ programming} focusses on how a problem is to be solved as a list of \idx[statement]{statements} and and a set of \idx{states}, where states may change over time. An example is a baking recipe, e.g., 
  \begin{enumerate}
  \item Mix yeast with water 
  \item Stir in salt, oil, and flour 
  \item Knead until the dough has a smooth surface 
  \item Let the dough rise until it has double size 
  \item Shape dough into a loaf 
  \item Let the loaf rise until double size 
  \item Bake in oven until the bread is golden brown
  \end{enumerate}
Each line in this example consists of one or more statements that are to be executed, and while executing them states such as size of the dough, color of the bread changes, and some execution will halt execution until certain conditions of these states are fulfilled, e.g., the bread will not be put into the oven for baking before it has risen sufficiently.

Statements may be grouped into procedures, and structuring imperative programs heavily into procedures is called \idx{Procedural programming}, which is sometimes considered as a separate paradigm from imperative programming. \idx[object-oriented programming]{Object-oriented programming} is an extension of imperative programming, where statements and states are grouped into classes and will be treated elsewhere.

Almost all computer hardware is designed for \idx{machine code}, which is a common term used for many low-level computer programming language, and almost all machine langues follow the imperative programming paradigm. 

  \idx{Functional programming} may be considered a subset of imperative programming, in the sense that functional programming does not include the concept of a state, or one may think of functional programming as only have one unchanging state. Functional programming has also a bigger focus on what should be solved, by declaring rules but not explicitly listing statements describing how these rules should be combined and executed in order to solve a given problem. Functional programming will be treated elsewhere.


\section{Generating random texts}
\subsection{0'th order statistics}
\fs{randomTextOrder0}{}

\subsection{1'th order statistics}
\fs{randomTextOrder1}{}
\end{document}
%%% Local Variables:
%%% TeX-master: "fsharpNotes"
%%% End:
