\chapter{The Event-driven Programming Paradigm}
\label{chap:eventDriven}

In \idx{event-driven programming}, the flow of the program is determined by \idx{events}, such as the user moving the mouse, an alarm going off, a message arriving from another program, or an exception being thrown, and is very common for programs with extensive interaction with a user, such as a graphical user interface. The events are monitored by \idx{listeners}, and the programmer can set \idx{handlers} which are \idx{call-back} functions to be executed when an event occurs. In event-driven programs, there is almost always a main loop to which the program relinquishes control to when all handlers have been set up. Event-driven programs can be difficult to test, since they often rely on difficult-to-automate mechanisms for triggering events, e.g., testing a graphical user interface often requires users to point-and-click, which is very slow compared to automatic unit testing.

%%% Local Variables:
%%% TeX-master: "fsharpNotes"
%%% End:
