\documentclass{article}
\usepackage{listings}

\begin{document}
\noindent Here is an example of gobbling spaces after quotes in lstinline. I am testing the combinations of with and without '[b]"' and ttfamily fonts:

\begin{description}
\item[No arguments: 2nd space is removed]~\\
  \begin{description}
  \item[0:] \lstinline !printfn "Test 1 2 \%d"3!
  \item[1:] \lstinline !printfn "Test 1 2 \%d" 3!
  \item[2:] \lstinline !printfn "Test 1 2 \%d"  3!
  \item[3:] \lstinline !printfn "Test 1 2 \%d"   3!
  \end{description}
\item[{[b]"}: 1st space is removed]~\\
  \begin{description}
  \item[0:] \lstinline[morestring={[b]"}] !printfn "Test 1 2 \%d"3!
  \item[1:] \lstinline[morestring={[b]"}] !printfn "Test 1 2 \%d" 3!
  \item[2:] \lstinline[morestring={[b]"}] !printfn "Test 1 2 \%d"  3!
  \item[3:] \lstinline[morestring={[b]"}] !printfn "Test 1 2 \%d"   3!
  \end{description}
\item[ttfamily: more than 1 space removed]~\\
   \begin{description}
  \item[0:] \lstinline[basicstyle=\ttfamily] !printfn "Test 1 2 \%d"3!
  \item[1:] \lstinline[basicstyle=\ttfamily] !printfn "Test 1 2 \%d" 3!
  \item[2:] \lstinline[basicstyle=\ttfamily] !printfn "Test 1 2 \%d"  3!
  \item[3:] \lstinline[basicstyle=\ttfamily] !printfn "Test 1 2 \%d"   3!
  \end{description}
\item[{[b]"} and ttfamily - first 2 spaces are removed]~\\
   \begin{description}
  \item[0:] \lstinline[morestring={[b]"},basicstyle=\ttfamily] !printfn "Test 1 2 \%d"3!
  \item[1:] \lstinline[morestring={[b]"},basicstyle=\ttfamily] !printfn "Test 1 2 \%d" 3!
  \item[2:] \lstinline[morestring={[b]"},basicstyle=\ttfamily] !printfn "Test 1 2 \%d"  3!
  \item[3:] \lstinline[morestring={[b]"},basicstyle=\ttfamily] !printfn "Test 1 2 \%d"   3!
  \end{description}
\end{description}
Which is different when using begin-end-lstlisting, no arguments:
\begin{lstlisting}
printfn "Test 1 2 \%d"3
printfn "Test 1 2 \%d" 3
printfn "Test 1 2 \%d"  3 
printfn "Test 1 2 \%d"   3 
\end{lstlisting}
\end{document}
